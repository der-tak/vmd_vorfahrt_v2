\chapter{Public Methods}
\label{cha:public_meth}
%
In diesem Kapitel werden alle public methods der wichtigsten Klassen genannt und ihre Funktion erläutert.
%
\section{\ssc}
\label{sec:statemeth}
%
\textbf{public string GetLanguage()}\\
gibt die aktuell gewählte Sprache zurück\\
\\
\textbf{public void SetLanguage(string language)}\\
aktualisiert die ausgewählte Sprache\\
\\
\textbf{public void SetCurScene()}\\
ermittelt die aktuelle Scene auf Basis von Unitys Scene Index\\
\\
\textbf{public int GetSelectedVehicle()}\\
gibt den Listen Index des aktuell ausgewählten Fahrzeugs zurück\\
\\
\textbf{public void SetSelectedVehicle(int VehicleNumber)}\\
aktualisiert den Index des ausgewählten Fahrzeugs\\
\\
\textbf{public void SetSelectedSide(string jsonData)}\\
setzt die ausgewählte Seite abhängig von settings.json im \sad \\
0 = left | 1 = right\\
\\
\textbf{public int GetLoadedSide()}\\
gibt die ausgewählte Seite als Integer zurück\\
0 = left | 1 = right\\
\\
\textbf{public string GetSelectedSide()}\\
gibt die ausgewählte Seite als String (left | right) zurück\\
%
\section{\vhc}
\label{sec:vhmeth}
%
\textbf{public Vehicle(string name)}\\
Constructor der \vhc:\\
String Parameter wird als \enquote{name}-Attribute an die neu erzeugte \vhi übergeben\\
ABER: nicht verwechseln mit dem Titel des Fahrzeugs\\
\\
\textbf{public void LoadText(string jsonData)}\\
bekommt einen rohen json-String übergeben und ordnet die json-Attribute den identischen Variablen der Klasse zu\\
\\
\textbf{public string GetName()}\\
gibt den Namen des Fahrzeugs zurück\\
ABER: nicht verwechseln mit dem Titel des Fahrzeugs\\
\\
\textbf{public string GetGerTitle()}\\
gibt den vollständigen Titel des Fahrzeugs auf deutsch zurück\\
\\
\textbf{public string GetEngTitle()}\\
gibt den vollständigen Titel des Fahrzeugs auf englisch zurück\\
\\
\textbf{public string GetYear()}\\
gibt das Erscheinungsjahr des Fahrzeugs zurück\\
\\
\textbf{public string GetGerHeader()}\\
gibt den Header (Titel + Jahr kombiniert) des Fahrzeugs auf deutsch zurück\\
\\
\textbf{public string GetEngHeader()}\\
gibt den Header (Titel + Jahr kombiniert) des Fahrzeugs auf englisch zurück\\
\\
\textbf{public string GetGerPreDescr()}\\
gibt die technischen Eckdaten des Fahrzeugs auf deutsch zurück\\
\\
\textbf{public string GetEngPreDescr()}\\
gibt die technischen Eckdaten des Fahrzeugs auf englisch zurück\\
\\
\textbf{public string GetGerDescr()}\\
gibt die Beschreibung des Fahrzeugs auf deutsch zurück\\
\\
\textbf{public string GetEngDescr()}\\
gibt die Beschreibung des Fahrzeugs auf englisch zurück\\
\\
\textbf{public Texture2D GetTitlePic()}\\
gibt das Titelbild des Fahrzeugs als Texture2D zurück\\
\\
\textbf{public void SetTitlePic(Texture2D titlePic, string titlePicName)}\\
setzt bzw. überschreibt das Titelbild und dessen Bezeichnung\\
\\
\textbf{public void SetMagazine(Texture2D loadedGallery, string vehicleName)}\\
fügt der Galeriebilder-Liste einen Eintrag hinzu und weist diesem einen Namen zu\\
\\
\textbf{public List<Texture2D> GetGallery()}\\
gibt die Galeriebilder als Liste vom Typ Texture2D zurück\\
\\
\textbf{public Texture2D GetGallery(int index)}\\
gibt ein einzelnes Galeriebild auf Basis das übergebenen Index zurück\\
\\
\textbf{public void Set3DModel(GameObject vehiclePrefFab)}\\
fügt das 3D-Model des Fahrzeugs der Instanz hinzu\\
\\
\textbf{public GameObject Get3DModel()}\\
gibt das 3D-Model als GameObject zurück\\
%
\section{\lsc}
\label{sec:lsc}
%
\textbf{public void SwitchLanguage()}\\
wechselt die aktuelle Sprache von deutsch zu englisch bzw. von englisch zu deutsch\\
%
\section{\ctlsc}
\label{sec:csc}
%
\textbf{public void ControlSwitcher()}\\
wechselt die aktiven Steuerungs-Scripte zwischen der Vehicle-Ansicht und der 3D-Vollbildansicht\\
%
\section{\doc}
%
\textbf{public void DisableText()}\\
deaktiviert nicht länger benötigte Objekte der Text-Vollbildansicht\\
wird in diesem Fall am Ende einer Animation gestartet\\
\\
\textbf{public void Disable3D()}\\
deaktiviert nicht länger benötigte Objekte der 3D-Vollbildansicht\\
wird in diesem Fall am Ende einer Animation gestartet\\
%
\section{\slc}
%
\textbf{public void LoadNextScene()}\\
erhöht den Scene Index um 1 und wechselt damit zur nächsten Scene\\
\\
\textbf{public void LoadNextScene(int sceneID)}\\
hier wird der Scene Index direkt übergeben, um eine bestimmte Scene zu laden\\
\\
\textbf{public void CheckPreloadScene()}\\
prüft anhand des \emph{\_\_app}-GameObject ob die \pres{} bereits gestartet wurde oder nicht\\
nur während der Entwicklung relevant, das fertige Build startet immer von der \pres{}\\
\\
\textbf{public void LoadMenuScene()}\\
lädt direkt in die \mms\\
\\
\textbf{public void LoadVehicleScene()}\\
lädt direkt in die \vhs\\
\\
\textbf{public void quit()}\\
schließt das Programm
%
\section{\mec}
%
\textbf{public void SetYear(string importedYear)}\\
weist dem entsprechenden Textfeld im MenuEntry-Prefab das Erscheinungsjahr zu\\
wird von der \mms{} \ref{sec:mms} aufgerufen, um die Einträge zuzuweisen\\
\\
\textbf{public void Setname(string importedName)}\\
weist dem entsprechenden Textfeld im MenuEntry-Prefab den Titel zu\\
wird von der \mms{} \ref{sec:mms} aufgerufen, um die Einträge zuzuweisen\\
%
\section{\msc}
%
\textbf{public void UpdateSaenfte()}\\
aktualisiert den Titel der Sänfte in der \mms{} auf deutsch bzw. englisch\\
\\
\textbf{public void UpdatePennyFarth()}\\
aktualisiert den Titel des Hochrads in der \mms{} auf deutsch bzw. englisch\\
\\
\textbf{public void SwitchVehicle(int num)}\\
aktualisiert das ausgewählte Fahrzeug im \ssc\\
%
\section{\orc}
%
\textbf{public void ToggleCover()}\\
toggelt die Animation der Motorhaube des Phänomen 4RL Modells\\
\\
\textbf{public void ToggleDriverDoor()}\\
toggelt die Animation der Fahrertür des Phänomen 4RL Modells\\
\\
\textbf{public void ToggleCoDoor()}\\
toggelt die Animation der Beifahrertür des Phänomen 4RL Modells\\
%
\section{\ars}
%
\textbf{public void disableInput()}\\
deaktiviert das \ars\\
\\
\textbf{public void enableInput()}\\
aktiviert das \ars\\
%
\section{\srs}
%
\textbf{public void disableInput()}\\
deaktiviert das \srs\\
\\
\textbf{public void enableInput()}\\
aktiviert das \srs\\
%
\section{\zcs}
%
\textbf{public void SliderZoom(float zoomValue)}\\
verschiebt die Kameraposition entlang der ihrer lokalen Z-Achse und den FOV-Wert um den übergebenen Float-Wert\\
kommt beim Zoom-Slider zum Einsatz\\
\\
\textbf{public void ResetCam()}\\
setzt die Kameraposition und den FOV-Wert auf den Standard-Wert zurück
%
\chapter{Einführung}
\label{chap:intro}
Im \vmd{} Dresden läuft eine Dauerausstellung unter dem Titel \enquote{Straßenverkehr}. Ein Großteil der Fahrzeuge dieser Ausstellung sind auf einer separaten Empore platziert. Diese ist jedoch für Besucher nur über vereinzelte Zugänge erreichbar und somit sind viele der Fahrzeuge inklusive ihrer Texttafeln nicht sichtbar. Um diesem Zustand entgegenzuwirken, setzt das \vmd{} im Besucherbereich der Empore zwei Tablets ein, auf denen eine \mapp{} installiert ist. Sie dient zur Visualisierung der schwer einsehbaren Bereiche der Ausstellung. Die App wurde ursprünglich von den Studenten Paul Wolff, der die Programmierung übernommen hat, und Johann Ludwig, der für die Erstellung der 3D-Modelle zuständig war, erstellt. Dafür wurde ein Programm auf Basis der \unite{} geschrieben. Es bietet eine Ansicht zur Auswahl der Fahrzeuge, Detailansichten sowie weitere Bild- und Textinformationen zum jeweils ausgewählten Fahrzeug. Das Museum stellte zu diesem Zweck eingescannte Archivaufnahmen und die Beschreibungstexte in deutscher und englischer Version zur Verfügung.
%
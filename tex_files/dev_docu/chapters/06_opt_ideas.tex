\chapter{Optimierungs Ideen}
\label{cha:opt}
%
Hier werden noch einige Optimierungs-Ideen aufgezählt, die während der Entwicklung nicht implementiert werden konnten. Diese Features waren entweder nicht zwingend erforderlich oder hätten zu viel Zeit für eine (aus Nutzer Sicht) geringe Verbesserung in Anspruch genommen. 
%
\section{Vollständig Dynamisches Laden}
\label{sec:full_dyn}
%
Für den Ladeprozess verwendet das \dls{} aktuell (Stand 05.08.21) noch eine zuvor vom Entwickler erstellte Liste, um die einzelnen Ordner des \sad{} zu durchsuchen. Hier wäre eine Methode interessant, die es dem Script ermöglicht die Ordner unter \path{/StreamingAssets/left/}bzw.  \path{/StreamingAssets/rechts/} selbstständig zu in eine Liste zu laden ähnlich, wie es in Zeile 3 \ref{lst:dyn_loader} zu sehen ist. Die \emph{getFiles()}-Methode der \emph{FileInfo}-Class kann eine Liste gefundener Ordner in einem Verzeichnis zurückgeben. Da diese Methode allerdings eine Systemeigene C\#- und keine Unity-Methode ist, steht sie im finalen Build nicht mehr zur Verfügung, wodurch der gesamte Ladevorgang außerhalb des \uedit s nicht mehr funktioniert. Es müsste also an dieser Stelle eine Unity-Methode gefunden werden, die diese Funktion ebenfalls mitbringt und als Unity eigene Funktion auch im finalen Build noch funktioniert. Außerdem müsste die Unity-Methode mit dem \uwr -Ladesystem kompatibel sein, da dieses für das Laden aus dem Streaming-Assets Ordner genutzt wird. 
%
\lstinputlisting[language={[Sharp]C},
caption={Die \emph{FileInfo}-Class ist dazu in der Lage Verzeichnisse zu durchsuchen und alle gefunden Ordner als Liste zurückzugeben.}, firstline=112, lastline=116, label={lst:dyn_loader}] {../../../coding/unity_projects/vmd_vorfahrt_v2/Assets/Scripts/DataLoader.cs}
%
\pagebreak
%
\section{Vermeiden von Hardcoded Delays}
\label{avoid_delays}
%
Die Ladevorgänge im \dls{} nutzen an vielen Stellen den Befehl \emph{yield return}, um den darauf folgenden Code auf mehrere Threads aufzuteilen (Zeile 7) \ref{lst:thread_spread}. Das hat den Vorteil, dass diese Teile des Codes, besonders das Laden der Bilder, parallel verarbeitet werden kann. Ab diesem Punkt wird der Code somit nicht mehr linear abgearbeitet, was die Ladegeschwindigkeit bei Multicore-CPUs optimiert allerdings auch dafür sorgt, dass das Script selbst keine Information mehr hat, wann der ausgelagerte Ladevorgang abgeschlossen ist. Der aktuell genutzte \enquote{quick-fix} für dieses Problem ist ein hardcoded Delay, das während der \sm{} ausgeführt wird (Zeile 10) \ref{lst:hard_delay}. Diese Methode ist eher ein Workaround statt eine tatsächliche Lösung. An dieser Stelle wäre eine Funktion erforderlich, die prüfen kann, ob aktuell noch ausgelagerte Funktionen arbeiten oder ob bereits alle Ladevorgänge abgeschlossen sind.  
%
\lstinputlisting[language={[Sharp]C},
caption={Code Teile, die auf \emph{yield return} folgen, können parallel verarbeitet werden.}, firstline=214, lastline=231, label={lst:thread_spread}] {../../../coding/unity_projects/vmd_vorfahrt_v2/Assets/Scripts/DataLoader.cs}
%
\lstinputlisting[language={[Sharp]C},
caption={Das hardcoded Delay fon einer Sekunde stellt vor dem Szenenwechsel sicher, dass alle Ladevorgänge abgeschlossen sind.}, firstline=262, lastline=274, label={lst:hard_delay}] {../../../coding/unity_projects/vmd_vorfahrt_v2/Assets/Scripts/DataLoader.cs}
%
\section{Dynamisches Laden der 3D-Modelle}
\label{sec:dyn_3d}
%
Das \dls{} greift aktuell auf die Daten (Texte und Bilder) im \sad zu, um diese bei jedem Programmstart neu einzulesen. Somit kann man theoretisch neue Fahrzeug-Einträge erstellen, ohne dafür einen neuen Build erstellen zu müssen. Wie bereits in \ref{sec:full_dyn} erwähnt, ist das bisher durch die Limitierung im \dls{} bisher ohnehin nicht implementiert. Ein weiteres Problem stellt das Laden der 3D-Modelle dar. Die verwendeten .fbx Dateien lassen sich nicht mit Hilfe des \uwr -Systems zur Laufzeit laden. Eine möglich Methode, die zunächst getestet werden müsste, wäre das Laden von .GLTF-Dateien, wie es in diesem Artikel beschrieben ist:\\ \url{https://theslidefactory.com/loading-3d-models-from-the-web-at-runtime-in-unity/}.
%
\section{Raycast basierte Animation-Trigger}
%
Die hervorgehobenen Teile des \emph{Phänomen 4RL} können mit einem Klick/Berührung animiert werden. Für diese unmittelbare Interaktion kommt ein Raycast-System zum Einsatz. Von dem Punkt aus, an der der Display angeklickt/berührt wurde, wird ein Strahl in die Szene geschickt. Sollte dieser Strahl auf ein animierbares Objekt treffen, wird die Animation getriggert und entweder vorwärts oder rückwärts abgespielt. Jedoch funktioniert dieses System derzeit (Stand 17.08.21) nur auf einer 1080p Auflösung präzise. sobald sich die Auflösung ändert, kommt ein Offset hinzu dessen Ursache noch nicht gefunden wurde. Da der Einsatz der App aber zu diesem Zeitpunkt lediglich auf Geräten geplant ist, die über eine 1080p Auflösung verfügen, sollte diese Einschränkung keine Probleme verursachen.
%
\section{Import weiterer Animationen}
\label{sec:imp_anim}
%
Das Phänomen 4RL ist zum jetzigen Stand (05.08.21) das einzige Fahrzeug, das über Animationen verfügt. Diese Animationen werde über Unitys internes Animationssystem gesteuert. Dieses System bringt allerdings für jegliche neue Animationen einen sehr hohen Setup-Overhead mit sich und sollte deshalb für weitere Animationen vermieden werden. Das 3D-Model selbst bringt beim Import in Unity aktuell keinerlei Informationen zu den Animationen mit. Es müsste also innerhalb des 3D-Programms ein Rig und entsprechende Animationen für jedes Fahrzeug erstellt werden. Auf diese Weise könnten die voreingestellten Animationen direkt beim Import in Unity ausgelesen werden und über entsprechende Trigger abgespielt werden.
%
\section{Touchinput beschränken}
\label{sec:limit_touch}
%
Der Touchinput und sämtlich Scroll-Fields verwenden von Unity aus die gleichen Input Befehle. Das führt dazu, dass beim Hoch-bzw. Runterscrollen eines Textfelds oft auch das 3D-Modell mit hin und her rotiert wird. In der aktuellen 
Version der Software (Stand 05.08.21) wird dieses Problem teilweise gelöst, in dem die Rotationssteuerung deaktiviert wird, sobald ein Scroll-Field betätigt wird und damit neue Werte liefert. das sorgt allerdings dafür, dass bei stillstehendem Scroll-Field weiterhin durch leichte links-rechts-Bewegungen eine Rotation des Modells getriggert werden kann. Eine deutlich sauberere Variante wäre es, die Touchinputs nur dann an die Rotationssteuerung zu senden, wenn diese innerhalb eines bestimmten Bereichs erfolgen.